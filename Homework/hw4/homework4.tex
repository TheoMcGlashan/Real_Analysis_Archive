\documentclass[12pt]{article}
\usepackage{mathtools}
\usepackage{amsthm}
\usepackage{amsfonts}
\usepackage{setspace}
\usepackage[margin=1in]{geometry}
\def\R{\mathbb{R}}
\def\N{\mathbb{N}}
\def\Z{\mathbb{Z}}
\def\Q{\mathbb{Q}}
\def\C{\mathbb{C}}
\title{Problem Set 4}
\author{Theo McGlashan}
\date{}
\onehalfspacing
\begin{document}
\maketitle
\newpage
\
\newpage

\subsection*{Aditional Problem 1}

\begin{itemize}
    \item[(a)] For $\emptyset \neq E \subseteq \R$ and $y \in \R$, define the distance of $y$ to $E$ as $$\text{dist}(y; E) := \inf_{x \in E} |x - y| \in [0, + \infty)$$
    
    $Claim.~$ For all $y, z \in \R$, one has 
    \begin{align}
        |\text{dist}(y; E) - \text{dist}(z; E)| \leq |y - z|. \label{eq:1}
    \end{align}
    Furthermore, this implies that the map $y \mapsto \text{dist}(y; E)$ is a continuous function from $\R$ to $\R$.

    \begin{proof}
        For the proof of (1), let $\epsilon > 0$. Then there exists $w \in E$ such that \begin{align}
            |w - z| - \epsilon < \text{dist}(y; E). \label{eq:2}
        \end{align}
        We can use this inequality to estimate the left side of \eqref{eq:1}:
        \begin{align}
            |\text{dist}(y;  E) - \text{dist}(z; E)| &= |\inf_{x \in E} |x - y| - \inf_{v \in E} |v - z| | \nonumber \\
            &< |\inf_{x \in E} |x - y| - (|w - z| - \epsilon) | \label{eq:3}\\
            &\leq ||w - y| - |w - z| | + \epsilon \label{eq:4} \\
            &\leq |w-y-w+z+\epsilon| \nonumber \\
            &\leq |y - z| + \epsilon \nonumber
        \end{align}
        where \eqref{eq:3} holds because of \eqref{eq:2} and \eqref{eq:4} holds by the definition of the infimum. Because the inequality in \eqref{eq:3} is strict, we can remove the $\epsilon$ and our proof of \eqref{eq:1} is complete.

        To show continuity of the map $y \mapsto \text{dist}(y; E)$, let $\epsilon > 0$. Then let $\delta = \epsilon$. Then for all $x \in E$ where $|x - y| < \epsilon$, we know by \eqref{eq:1} that $$|\text{dist}(y; E) - \text{dist}(x, E)| \leq |y - z| < \delta = \epsilon.$$ Therefore this map is continuous.
    \end{proof}

    \newpage

    \item[(b)]
    
    $Claim.~$ Assume $\epsilon \neq \emptyset$ is closed. Then dist$(y; E) = 0$ if and only if $y \in E$.

    \begin{proof}
        For the forward direction of this claim, assume that dist$(y; E) = 0$. Then $\inf_{x \in E} |x - y| = 0$, and for all $\epsilon > 0$, there exists $w \in E$ such that $$| \inf_{x \in E} |x - y| - |w - y| | < \epsilon.$$ Then let $\epsilon = \frac{1}{n}$, and use this statement to construct the sequence $(W_n)_{n \in \N}$ where for all $n \in \N$, the statement holds for $w_n$. Then $\lim_{n \to \infty} w_n = y$. But $E$ is a closed set, so it contains its limit points, and $w_n \in E$ for all $n \in \N$, so $y \in E$.

        For the backward direction of the claim, assume that $y \in E$. Then dist$(y; E) = \inf_{x \in E} |x - y|$, but dist$(y; E) \geq 0$, and $|y - y| = 0$, so $\inf_{x \in E} |x - y| = 0$.
    \end{proof}

    For a counterexample when $E$ is not closed, let $E = (0,1)$, and let $y = 0$. Then for $n \in \N$, let $\epsilon = \frac{1}{2n}$. We know that for all $n \in \N$, we have $$y + \frac{1}{2n} = \frac{1}{2n} \in (0,1) = E.$$ Therefore $\inf_{x \in E} |x - y| \leq \frac{1}{2n}$ for all $n \in \N$, so $\inf_{x \in E} |x - y| = 0$, meaning dist$(y; E) = 0$ for $y \notin E$, completing the counterexample.

    \item[(c)]
        $Urysohn's~Lemma.$ Let $E, F$ be two \textit{non-empty, disjoint, and closed} sets in $\R$. Then, there exists a continuous function $g : \R \to \R$ with $0 \leq g \leq 1$ such that $g = 0$ on $E$ and $g = 1$ on $F$.

        \begin{proof}
            Let $E, F$ be as described in the lemma and let $$g(y) := \frac{\text{dist}(y; E)}{\text{dist}(y; E) + \text{dist} (y; F)}, ~y \in \R.$$
            
            First, assume $y \in E$. Then dist$(y; E) = 0$ by (b), and because $E$ and $F$ are disjoint, $y \notin F$, so again by (b) we know dist$(y; F) > 0$. Therefore $g(y) = 0$.

            Then assume $y \in F$. Then dist$(y; F) = 0$ and dist$(y; E) > 0$, so $$g(y) = \frac{\text{dist} (y; E)}{\text{dist}(y; E)} = 1.$$

            Finally, assume $y \notin E$ and $y \notin F$. Then both dist$(y; E)$ and dist$(y; F)$ are positive, so $0 < g(x) < 1$.
        \end{proof}
\end{itemize}

\subsection*{Additional Problem 2}

For $\mu_\N$, the counting measure of the natural numbers, let $(\R, \mathcal{P}, \mu_\N)$ be a measure space. Additionally, let $0 \leq f: \R \to \R$ be a function.

\begin{itemize}
    \item[(a)] $Claim.~$ The statement
    \begin{align}
        \int f d\mu_\N = \sum_{n = 1}^\infty f(n) \label{eq:5}
    \end{align}
    holds for all finitely supported, non-negative, simple functions.

    \begin{proof}
        Because $f$ is a simple function, it can be written as $\sum_{k=1}^N c_k \chi_{E_k}$ for some $c_1, \ldots c_N \in \R$ and $E_1, \ldots E_N \subseteq \R$. Then by the definition of the integral of simple functions, we know $$\int f d\mu_\N = \sum_{k=1}^N c_k \mu_\N (E_k).$$ Now looking at the right side of \eqref{eq:5}, observe that for any $n \in N$, $$f(n) = \begin{cases}
            c_k &\text{ if } x \in E_k \\
            0 &\text{ if } x \notin E_k \text{ for all k}
        \end{cases}$$
        Therefore for any $c_k$, the sum on the right side of \eqref{eq:5} adds that $c_k$ term for each $n \in E_k$, where $n \in N$. But this is precisely $c_k \mu_\N (E_k)$, so $$\sum_{n=1}^{\infty} f(n) = \sum_{n=1}^{N} c_k \mu_\N (E_k) = \int f d \mu_\N.$$
    \end{proof}

    \item[(b)]
    
    $Lemma.~$ \textit{Given a measure space} $(Y, \mathcal{S}, \mu)$, \textit{suppose that} $(Y_n)_{n \in N}$ \textit{is an} $\mathcal{S}$\textit{-valued sequence of nested increasing sets such that} $$Y = \bigcup_{n \in \N} Y_n.$$ \textit{Then for every measurable function} $0 \leq f : Y \to \R$, \textit{one has} $$\lim_{n \to \infty} \int \chi_{Y_n} f d\mu = \int f d \mu$$
    
    $Claim.~$ Equation \eqref{eq:5} holds for an arbitrary function $0 \leq f: \R \to \R$.

    \begin{proof}
        Let $(Y_n)_{n \in \N}$ be a sequence of nested increasing sets defined by $Y_n = [-n, n]$.
        Then observe that $\mu_\N$-a.e. , we know that $$\chi_{Y_n} f = \sum_{k=1}^{n} f(k) \chi_{\{k\}} =: \varphi_n.$$ Because these differ on a set of finite measure only, we know $$\int \chi_{y_n} f d \mu_\N = \int \sum_{k=1}^{n} f(k) \chi_{\{k\}} d \mu_\N.$$ But $\varphi_n$ is a simple function, so by (a), we have $$\int f \chi_{Y_n} d \mu_\N = \int \varphi d \mu_\N = \sum_{k=1}^\infty \varphi_n (k).$$ Then because $\lim_{n \to \infty} Y_n = Y$, we can say $$\int f d \mu_\N = \lim_{n \to \infty} \int f \chi_{Y_n} d \mu_\N = \lim_{n \to \infty} \sum_{k=1}^{\infty} \varphi_n (k) = \sum_{k=1}^{\infty} f(k)$$
    \end{proof}
    \item[(c)]
    $Proof~of~Lemma.~$ Let $(f_n)_{n \in N}$ be a series of functions defined by $f_n = \chi_{Y_n} f$. Then $(f_n)_{n \in \N}$ is an increasing sequence of measurable functions. We also know $\lim_{n \to \infty} f_n = f$, so we can apply the monotone convergence theorem to get $$ \lim_{n \to \infty} \int \chi_{Y_n} f d \mu = \int f d \mu.$$
\end{itemize}

\subsection*{Additional Problem 3}

A Borel measure $\mu$ on $X \subseteq \R$ is regular if for every Borel set $A \subseteq X$, there exists a closed set $f \subseteq A$ and open set $O \supseteq A$ such that $$\mu(A \setminus F) < \epsilon \text{ and } \mu(O \setminus A) < \epsilon$$
\begin{itemize}
    \item[(a)] Let $\mu$ be a finite regular Borel measure on $X = [a, b]$, and $f : [a,b] \to \R$ be a measurable function.
    
    $Claim.~$ For all $\epsilon, \eta > 0$, there exists a step function $s : [a, b] \to \R$ such that $$ \mu( \{|f - s| > \eta\} ) < \epsilon.$$

    \begin{proof}
        Let $\epsilon, \eta > 0$. Then because $f$ is measurable, we know that there exists $A \subseteq [a, b]$ where $\mu([a, b] \setminus A) < \epsilon$ and $f|_A$ is bounded. Then construct a sequence of simple functions $$(\varphi_n)_{n \in \N} \text{ such that } \varphi_n \nearrow f \text{ on } A.$$ Then because $f$ is bounded on $A$, this convergence is uniform on $A$. Then there exists $N \in \N$ such that 
        \begin{align}
            | \varphi_N - f| < \eta. \label{eq:6}
        \end{align} By definition of simple functions, we know $$\varphi_n = \sum_{k=1}^{K_n} c_k \chi_{E_k} \text{ for } c_k \in \R \text{ and } E_k \subseteq \R.$$ Then because $f$ is defined on $[a, b]$ and $\mu ([a, b]) < \infty$ because $\mu$ is a finite measure, we know $\mu (E_k) < \infty$ for all $E_k$. Then by Littlewood's first principle, we have $$\mu \left( E_k ~\triangle \left( \bigsqcup_{m=1}^M I_{k, m}\right)\right) < \frac{\epsilon}{2^k} ~\text{ for intervals } I_{k,m}.$$ Then define a sequence of functions $(s_n)_{n \in N}$ where $$s_n = \sum_{k=1}^{K} c_k \sum_{m=1}^{M} \chi_{I_{k,m}}$$ assuming without loss of generality that all $I_{k, m}$ are disjoint. Then by this definition, $$\{|\varphi_n - s_n| \neq 0\} \subseteq \bigcup_{k=1}^K \left(E_k \triangle \bigsqcup_{m=1}^M I_{k, m}\right).$$ But the right side of the above has measure $\leq \sum_{k=1}^{\infty} \frac{\epsilon}{2^k} = \epsilon$, so the left side has measure $\leq \epsilon$, which is the same as measure $< \epsilon$ (just vary $\epsilon$ at the start of the proof). 

        Then for $s_N$, where $N$ is defined near the beginning of the proof, we have $$\mu( \{ |f - s_N| > \eta \}) <\mu( \{|f - \varphi_N | > \eta \}) + \epsilon = \epsilon.$$ Here the first inequality holds because the measure of the points where $\varphi_N$ and $s_N$ are different is less than $\epsilon$, and the second equality follows from \eqref{eq:6} telling us that there are no points where $|f - \varphi_N| > \eta$. Then because $S_N$ is a step function, we have our claim.
    \end{proof}

    \item[(b)] $Claim.~$ For every measurable function $f : [a, b] \to \R$ and every $\eta, \mu > 0$, there exists a continuous function $h [a, b] \to \R$ such that $$\mu( \{|f - h| > \eta \}) < \epsilon.$$
    
    \begin{proof}
        Let $\epsilon, \eta > 0$. Then by (a), there exists step function $s : [a, b] \to \R$ such that $\mu(\{|f - s| \eta\}) < \epsilon.$ Then by definition, $$s = \sum_{k=1}^{n} c_k \chi_{I_k} ~\text{ where } I_k \text{ are intervals and } c_k \in \R.$$ Assume without loss of generality that each $I_k$ is disjoint. Then for each $I_k$, we do not know if this is a closed or open interval. However, $\mu$ is regular, so we have a closed interval $I_k'$ such that $\mu(I_k \ I_k') < \epsilon$. Then for $I_j', I_{j+1}'$, we know by Urysohn's Lemma that there exists $g : \R \to \R$, a continuous function such that $0 \leq g \leq 1$ and $g|_{I_j'} = 0, ~g|_{I_{j+1}'} = 1$. Then for any $a \in I_j'$ and $b \in I_{j+1}'$, construct the function $$h_j := (s(a) - s(b))g + s(a)$$ which will be the same regardless of our choice of $x$ and $y$. This is a continuous function valued at $s(a)$ on $I_j'$ and $s(b)$ on $I_{j+1}'$, meaning it is valued at $s$ on these intervals. Then $h_j = h_{j+1}$ on $I_{j+1}'$, so we can define a function $h$ that takes the value of each $h_j$ for $I_j'$ to $I_{j+1}'$, then each $h_j$ will overlap with the next, but they will both have the same constant value where they overlap.

        Because of the construction of our intervals $I_k'$, we know that $h$ differs from $s$ at most on a set of measure $< n\epsilon$, which can still be made arbitrarily small, so our claim holds.
    \end{proof}

    \item[(c)] $Claim.~$ The counting measure $\mu_\N$ is sigma finite and regular on $\R$.
    
    \begin{proof}
        $\R = \bigcup_{n \in \N} (-n, n)$, and $\mu_\N (-n, n)$ is finite for all $n \in \N$, so $\mu_\N$ is sigma finite.

        For any Borel set $B$, construct an open set $O$ such that for each connected component of $B$, $O$ is locally the smallest open interval $(a, b)$ containing the component such that $a, b \in \N$. Then $\mu_\N (a, b)$ equals the measure of this component of $B$, so $\mu (O) = \mu (B)$.

        Then construct a closed set $F$ such that for each connected component of $B$, $F$ is locally the largest open interval $[c, d]$ contained in the component such that $c, d \in \N$. Then $\mu_\N [a, b]$ equals the measure of this component of $B$, so $\mu (F) = \mu (B)$. Therefore $\mu_\N$ is regular.
    \end{proof}

    $Claim.~$ The Dirac delta measure $\delta_c$ for every fixed $c \in \R$ is a regular Borel measure on $\R$.

    \begin{proof}
        Take $c \in \R$. Then for a Borel set $B$, construct an open set $O$. If $c \notin B$, then let $O$ be the intersection of all open sets containing $B$. Then if $c \in O$, then $c$ is in every open set containing $B$. But then $c \in B$, which is a contradiction, so $c \notin O$. Therefore $B$ and $O$ have measure $O$. If $c \in B$, then any open set containing $B$ has measure 1, as does $B$.

        Now construct a closed set $F$. If $c \in B$, then let $F$ be the union of all closed sets contained in $B$. Then if $c \notin F$, then $c$ is not in any closed set containing $B$. But then $c \notin B$, which is a contradiction, so $c \in F$. Therefore $B$ and $F$ both have measure 1. If $c \notin B$, then any closed set contained in $B$ has measure 0, as does $B$. Therefore this measure is regular on $\R$.
    \end{proof}
\end{itemize}

\subsection*{3A.3}

Suppose $(X, \mathcal{S}, \mu)$ is a measure space and $f : X \to [0, \infty]$ is an $\mathcal{S}$-measurable function.

$Claim.~$ $$\int f d \mu > 0 \text{ if and only if } \mu( \{ x \in X: f(x) > 0\}) > 0.$$

\begin{proof}
    ($\implies$) For the forward direction, we prove the contrapositive by showing that $$\mu( \{ x \in X: f(x) > 0\}) = 0 \implies \int f d \mu = 0.$$
    To do this, first define $B := \{ x \in X: f(x) > 0\}$. Then let $A_1, \ldots A_n$ be a partition of $\mathcal{S}$. Then $$\mathcal{L} (f; p) = \sum_{k=1}^{n} \mu (A_k) \inf_{A_k} f.$$ Then for any $A_k$ where $A_k \cap B \neq A_k$, there exists $x \in A_k$ such that $f(x) = 0$, so $\inf_{A_k} = 0$. Therefore only those $A_k$'s where $A_k \subseteq B$ are relevant for this sum. But $\mu(B) = 0$ by hypothesis, so $\mu(A_k) = 0$ for all $A_k \subseteq B$, so the whole sum is 0. Then because we chose a partition of $\mathcal{S}$ arbitrarily, we know $\int f d \mu = 0$.

    $(\impliedby$) For the backwards direction, let $(B_n)_{n \in \N}$ be a sequence of sets where $$B_n := \{x \in X : f(x) > \frac{1}{n}\}.$$ Then this sequence is nested increasing. so by our continuity properties we have $$\mu(\bigcup_{n \in \N} A_n) = \lim_{n \to \infty} \mu (A_n) > 0$$ where the last inequality follows from our hypothesis. Then assume that this limit is $c > 0$ for some $c \in \R$. Then by the definition of convergence, we know that there exists an $N \in \N$ such that $|\mu (A_N) - c| < c$. Then $\mu(A_n) > 0$. Now let $P = A_N, X \setminus A_N$ be a partition of $X$. Note $A_N \in \mathcal{S}$ because $A_N = f^{-1} (\frac{1}{N}, \infty)$. But then, noting $\inf_{A_N} f = \frac{1}{N}$, we have $$\int f d \mu \geq \mathcal{L} (f; P) = \mu (A_n) \inf_{A_N} f + \mu (X \setminus A_N) \inf_{X \setminus A_N} f \geq \mu (A_n) \frac{1}{N} > 0.$$
\end{proof}
\end{document}