\documentclass[12pt]{amsart}
\usepackage{amsmath}
\usepackage{amsthm}
\usepackage{amsfonts}
\usepackage{setspace}
\usepackage[margin=1in]{geometry}
\newcommand{\R}{\mathbb{R}}
\newcommand{\N}{\mathbb{N}}
\newcommand{\Z}{\mathbb{Z}}
\newcommand{\Q}{\mathbb{Q}}
\newcommand{\C}{\mathbb{C}}
\newcommand{\eq}[1]{\begin{equation*}#1\end{equation*}}
\newcommand{\al}[1]{\begin{align*}#1\end{align*}}
\newcommand{\qeq}[1]{\begin{equation}#1\end{equation}}
\newcommand{\qal}[1]{\begin{align}#1\end{align}}
\title{Keyword Document}
\author{Theo McGlashan}
\date{}
\onehalfspacing
\begin{document}
\maketitle

\subsection*{Key Result 1: Parseval's Identity}

Suppose $\{e_k\}_{k \in \Gamma}$ is an Orthonormal basis of a hilbert space $V$ and $f, g \in V$. Then
\begin{itemize}
    \item [(a)] $f = \sum_{k \in \Gamma} \langle f, e_k \rangle e_k$;
    \item [(b)] $\langle f, g \rangle = \sum_{k \in \Gamma} \langle f, e_k \rangle \overline{\langle g, e_k \rangle}$;
    \item [(c)] $\|f\|^2 = \sum_{k \in \Gamma} | \langle f, e_k \rangle |^2$.
\end{itemize}



\end{document}