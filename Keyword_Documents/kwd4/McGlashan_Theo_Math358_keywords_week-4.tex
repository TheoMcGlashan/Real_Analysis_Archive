\documentclass[12pt]{article}
\usepackage{mathtools}
\usepackage{amsthm}
\usepackage{amsfonts}
\usepackage{setspace}
\usepackage[margin=1in]{geometry}
\def\R{\mathbb{R}}
\def\N{\mathbb{N}}
\def\Z{\mathbb{Z}}
\def\Q{\mathbb{Q}}
\def\C{\mathbb{C}}
\title{Keyword Document 4}
\author{Theo McGlashan}
\date{}
\onehalfspacing
\begin{document}
\maketitle

Reading: Chapter 3, p. 82-84, 88-98.

\subsection*{Key Result 1: Dominated Convergence Theorem}

Suppose ($X, \mathcal{S}, \mu$) is a measure space, $f : X \to [-\infty, \infty]$ is $\mathcal{S}$-measurable, and $f_1, f_2, \ldots$ are $\mathcal{S}$- measurable functions from $X$ to $[-\infty, \infty]$ such that $$\lim_{k \to \infty} f_k(x) = f(x)$$ for almost every $x \in X$. If there exists an $\mathcal{S}$-measurable function $g : X \to [0, \infty]$ such that $$\int g ~ d \mu < \infty ~\text{ and }~ |f_k(x)| \leq g(x)$$ for every $k \in \N$ and almost every $x \in X$, then $$\lim_{k\to\infty} \int f_k ~d \mu = \int f ~d\mu.$$

This is our second major theorem, the first being the Monotone Convergence Theorem, that allows us to interchange limits and integrals. This ability to exchange limits and integrals is one of the most important properties for proving theorems, as when we approximate functions with series of simple functions, we can then say meaningful things about the integral of the function being approximated. This theorem gives us another set of hypotheses where we can do this. This result is on page 92.

\subsection*{Key Result 2: Properties of the $\mathcal{L}^1$-norm}

Suppose $(X, \mathcal{S}, \mu)$ is a measure space and $f, g \in \mathcal{L}^1$. Then
\begin{itemize}
    \item $||f||_1 \geq 0$;
    \item $||f||_1 = 0$ if and only if $f(x) = 0$ for almost every $x \in X$;
    \item $||cf||_1 = c||f||_1$ for all $c \in \R$;
    \item $||f + g||_1 \geq ||f||_1 + ||g||_1.$
\end{itemize}

This result gives us useful properties of the $\mathcal{L}^1$ norm. These properties fit our intuitive notion of what this norm should do, so it is good that they are indeed all true. All four of these properties are useful, but in particular, the last property gives us the triangle inequality for the $\mathcal{L}^1$-norm, which is very useful. This result is on page 96.

\subsection*{Key Technique}

3.24 gives us the definition on integrating on a subset of the domain of our function, which is used in several proofs after this. Interestingly, it is also implicitly used in proofs before this, like the proof for additivity of integration (3.21). Here the function is broken up into it's positive and negative components, and in integrating these components individually is equivalent to integrating the original function on the subsets of it's domain where it is positive and negative respectively. This technique is also used several times explicitly: in the proof for 3.26, 3.28, 3.29, and 3.31.
\end{document}